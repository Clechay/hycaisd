% !TEX encoding = UTF-8 Unicode
\documentclass[svgnames]{report}
\usepackage[utf8x]{inputenc} 
\usepackage{polski}       
\usepackage{a4wide}
\usepackage{graphicx}
\usepackage{amsmath,amssymb}
\usepackage{bbm}            % sudo apt-get install texlive-fonts-recommended texlive-fonts-extra
\usepackage{amsthm}
\usepackage{algorithmic}	% sudo apt-get install texlive-science
\usepackage{listings}             % Include the listings-package
\usepackage{framed}
\usepackage{enumerate}
\usepackage{hyperref}


\makeatletter
 \renewcommand\@seccntformat[1]{\csname  the#1\endcsname.\quad}
  

\lstset{language=C}
%\DeclareUnicodeCharacter{00A0}{ }

\begin{document}
%\chapter{lista 0}
%%%%%%%%%%%%%%%%%%%%%%%%%%%%%%%%%%%%%%%%%%%%%%%%%%%%%%%%%%%%%%%%%%%%
%\chapter{Lista 1}

\section{}
\section{}
\section{}
\section{}
\[a_1 = a, \; a_k = 1, \; a_{i+1} = \lfloor \frac{a_i}{2}\rfloor\]
\[b_a = b, \; b_{i+1} = 2b_i \Rightarrow b_i = 2^ib \] 
Niech $a = \sum_{i=1}^k 2^{i-1}\bar{a}_i$, $\bar{a}_i \in \{ 0,1\}$, czyli zapis w postaci binarnej. Wtedy

\[a_n = \sum_{i=1}^n 2^{i-1}\bar{a}_{i+(k-n)}\]

Dowód:

\[\sum_{i=1, \; nieparzyste \; a_i}^k b_i = \sum_{i=1}^k\bar{a}_i2^ib = b\sum_{i=1}^k2^i\bar{a}_i = ab\]

Kryterium jednorodne - koszt każdej operacji jest jednostkowy. Zatem ile jedynek w zapisie binarnym liczby $b$, taką mamy złożoność $\Rightarrow O(\lceil(log_2b)\rceil)$ \\
Kryterium lagarytmiczne - koszt operacji zależy od odługości operandów. Dodawań mamy tyle samo $\Rightarrow O(\lceil log_2b\rceil \cdot \lceil log_2 ab\rceil)$



\section{}

Niech $f_n = A_kf_{n-k} + A_{k-1}f_{n-k+1} + \ldots A_1f_{n-1}$. Jeżeli $f_n$ zależy od $k$ poprzednich lementów to tworzymy następującą macierz $k\times k$(x-kolumny, y-wiersze, indeksowanie od 0)

\[M_{x,y} = 1, \; dla \;x=y+1\]
\[M_{x,k} = A_{n- k + x}\]
Reszta elementów to 0. Przykład dla $k=3$
\[ 
 \left[
 	\begin{array}{ccc}
 	0		&	1		&	0\\
 	0		&	0		&	1\\
 	A_{n-3}	&	A_{n-2}	&	A_{n-1}
 	\end{array}
 \right]
\] 

Żeby policzyć $f_n$ tworzymy wektor $F = (f_{n-k} \ldots f_{n-1})$ i obliczamy $M\cdot F^T$.\newline


Tworzymy macierz rozmiaru $(k+m) \times (k+m)$, $k$ - ilość poprzednich wyrazów ciągu, $m$ - stopień wielomianu.
Dzielimy ją na cztery prostokąty. Lewy górny taki jak w poprzedniej części. Prawy górny wypełniamy zerami, oprócz ostatniego wiersza, który wypełniamy współczynnikami wielomianu. Lewy dolny wypełniamy zerami. Żeby wypełnić prawy dolny zastanówmy się najpierw jak uzyskać $(n+1)^i$. Oczywiście z dwumianu newtona. prawy dolny prostokąt będzie miał postać

\[
 \left[
 	\begin{array}{cccc}
 	{m \choose m}	&	\ldots				&	\ldots	& 	{m \choose 0}\\
 	0 				&	{m-1\choose m-1}	&	\ldots	&	{m-1 \choose 0}\\
 	\ldots			& 	\ldots 				& 	\ldots	& 	\ldots\\
 	0				&	0					&	\ldots	&	1
 	\end{array}
 \right]
\]


Znowu tworzymy sobie wektor $F = (f_{n-k} \ldots f_{n-1}, n^m, n^{m-1}, \dots, 1)$ i obliczamy $M\cdot F^T$.

\section{}


\section{}

Mamy daną listę $L$. Dzielimy ją na podlisty długości $\sqrt(n)$. Tworzymy dodatkową listę $K$ o długości $\sqrt(n)$, zawierającą wskaźniki na pierwsze elementy utworzonych wcześniej podlist. Przy wstawianiu elementu przeglądamy najpierw Listę $K$, a następnie listę $L$ od miejsca, na które wskazywał wskaźnik z $K$. Maksymalnie przejrzymy $2\sqrt(n)$ elementów. Po wstawieniu elementu musimy ouaktualnić listę wskaźników $K$, czego koszt to znowu $\sqrt(n)$

\section{}

Wejście: Skierowany acykliczny graf.\\
Wyjście: Długość najdłuższej ścieżki.\\
LengthTo - tablica $|V(G)|$ elementów początkowo równych 0.\\
TopOrder(G) - posortowane topologicznie wierzchołki.\\


\begin{lstlisting}
    for each vertex V in topOrder(G) do
        for each edge (V, W) in E(G) do
            if LengthTo[W] <= LengthTo[V] + weight(G,(V,W)) then
               LengthTo[W]  = LengthTo[V] + weight(G,(V,W))
 
    return max(LengthTo[V] for V in V(G))
\end{lstlisting}

Sortowanie topologiczne działa w czasie $O(E+V)$, więc całość działa w czasie $O(E+V+E+V) = O(E+V)$.
Żeby wypisać drogę musimy tylko zapamiętywać, dla których wierzchołków spełniony był IF.



\chapter{Lista 1.}

\section{}%1
\framebox[\textwidth]{Napisz rekurencyjne funkcje, które dla danego drzewa binarnego $T$ obliczają:}
\begin{itemize}
\item{liczbę wierzchołków $T$}
\begin{lstlisting}
def nodes tree
	return 0 if tree.nil?
	return 1 if tree.leaf?
	return 1 + nodes(tree.left) + nodes(tree.right)
\end{lstlisting}

\item{maksymalną odległość między wierzchołkami w $T$}
Jest to \textbf{średnica} drzewa.
Zauważmy, że:
\begin{itemize}
\item Puste drzewo ma średnicę $0$.
\item Jeśli drzewo jest niepuste, to przez $t_1$ i $t_2$ oznaczmy dwa poddrzewa zakorzenione w lewym i prawym synie korzenia. Odpowiednio przez $d_1$ i $d_2$ oznaczmy średnice tych poddrzew, a przez $h_1$ i $h_2$ ich wysokości. Wówczas średnica całego drzewa wynosi $max(d_1, d_2, h_1+h_2+2)$.
\end{itemize}
\begin{lstlisting}
def diameter(node) 
	return (-1,0) if node.nil? 
	(lheight, ldiameter) = diameter(node.left)
	(rheight, rdiameter) = diameter(node.right)
	
	height = max(lheight, rheight) + 1
	diameter = max(lheight + rheight + 2, ldiameter, rdiameter)

	return [height, diameter]
\end{lstlisting}
\end{itemize}
\section{}%2
\fbox{\parbox[t]{\textwidth}{Dla kopca minimaksowego. Przyjmij, że elementy pamiętane są w jednej tablicy (określ w jakiej kolejności). Napisz w pseudokodzie procedury: }}
\begin{itemize}
\item{przywracania porządku}
\begin{verbatim}
def level(i)
  floor(log2(i))

def min_level?(i)
  level(i)\%2 == 1

def trickle_down(K,i)
  if min_level?(i)
    trickle_down_min(K,i)
  else
    trickle_down_max(K,i)
  end

def trickle_down_min(K,i)
  return if K[k] has no children
  m = index of smallest of the children\
      and grandchildren (if any) of K[i]
  
  if K[m] is a child of K[i]
    if K[m] < K[i]
      swap(A[m], A[i])
  else # it grandchild from next min level
    if K[m] < K[i]
      swap(K[m],K[i])
      if K[m] < K[parent(m)]
        swap(K[m],K[parent(m)])
      trickle_down_min(K,m) //swap with parent doesn't change our min/max level
\end{verbatim}

W \textbf{trickle\_down} sprawdzamy czy element przesuwany w dół należy zamienić z którymś z dzieci oraz wnuków. Dodatkowo, element podmieniamy z wnukiem, sprawdzamy czy nie popsuliśmy ojca wnuka.

Jeśli podmieniliśmy element z bezpośrednim dzieckiem, to znaczy, że porządek jest zachowany i nie musimy schodzić niżej.

\begin{verbatim}
def bubble_up(K,i)
  if min_level?(i)
    if K[i] has a parent and
      K[i] > K[parent(i)]
      swap(K[i], K[parent(i)])
      bubble_up_max(K,parent(i))
    else
      bubble_up_min(K,i)
  else #max level
    if K[i] has a parent and
      K[i] < K[parent(i)]
      swap(K[i],K[parent(i)])
      bubble_up_min(K,parent(i))
    else
      bubble_up_max(K,i)

def bubble_up_min(K,i)
  if K[i] has a grandparent
    if K[i] < K[grandparent(i)]
      swap(K[i],K[grandparent(i)])
      bubble_up_min(K,grandparent(i))
\end{verbatim}
W \textbf{bubble\_up} sprawdzamy najpierw czy świeżo wstawiony element pasuje bardziej do poziomu, na który trafił przy wstawieniu, czy do poziomu wyżej (sprawdzamy czy jest pretendentem do maksimum czy minimum). Następnie przesuwamy go wyżej skacząc bo dziadkach, nie zaburzając przy tym porządku kopca.


\begin{verbatim}
def change(K,i,v)
  if min_level?(i)
    if K[i] < v
      K[i] = v
      trickle_down(K,i)
    else
      K[i] = v
      bubble_up(K,i)   
  else
    if K[i] > v
      K[i] = v
      trickle_down(K,i)
    else
      K[i] = v
      bubble_up(K,i)
\end{verbatim}

Jeżeli element pasuje lepiej na swoim poziomie niż poprzednik, to próbujemy upchnąć go wyżej (relacja z poniższymi elementami jest niezmieniona). Jeżeli element nie pasuje na swoim poziomie, musimy upchać go w dół.

\item{usuwania minimum}
\begin{verbatim}
def delete_min(K)
  min = K[1]
  swap(K[1],K[n])
  K = K[1..n-1]
  trickle_down(K,1)
  return min
 \end{verbatim}
\item{usuwania maksimum}
\begin{verbatim}
def delete_max(K)
  m = index of largest element 
      amongst K[1], K[2], K[3]
  max = K[m]
  swap(K[1], K[n])
  K = K[1..n-1]
  trickle_down(K,1)
  return max
\end{verbatim}
\end{itemize}
\section{}%3
\fbox{\parbox[t]{\textwidth}{
Porządkiem topologicznym wierzchołków acyklicznego digrafu $G = (V, E)$ nazywamy taki liniowy porządek jego wierzchołków, w którym początek każdej krawędzi występuje przed jej końcem. Jeżeli wierzchołki z $V$ utożsamimy z początkowymi liczbami naturalnymi to każdy ich przodek liniowy można opisać permutacją liczb $1,2,3,...,|V|$; w szczególności pozwala to na porównywanie leksykograficzne porządków.
Ułóż algorytm, który dla danego digrafu znajduje pierwszy leksykograficznie porządek.}}

$Q$ - Kolejka priorytetowa z wierzchołkami o stopniu wchodzącym równym $0$.


\begin{verbatim}
dopóki Q jest niepusta rób
    usuń wierzchołek n z przodu kolejki Q
    wypisz n
    dla każdego wierzchołka m o krawędzi e od n do m rób
        usuń krawędź e z grafu
        jeżeli do m nie prowadzi żadna krawędź to
            wstaw m do Q
jeżeli graf ma wierzchołki to
    wypisz komunikat o błędzie (graf zawiera cykl)
\end{verbatim}    

Jedyną zmianą w algorytmie sortowania topologicznego, jest podmienienie kolejki na kolejkę priorytetową. W ten sposób zawsze otrzymujemy wierzchołek spełniający warunek porządku topologicznego, który ma najmniejszy możliwy indeks.
Złożoność sortowania topologicznego wynosi $O(|E| + |V|)$. Ale w naszym przypadku, każde wstawienie elementu do kolejki trwa $O(log |V|)$ zamaist $O(1)$, zatem złożoność wynosi: $O(|E| + |V| log |V|)$.

\subsection{dowód poprawności}
Nasz algorytm jest zachłanny, stopniowo buduje rozwiązanie używając lokalnego kryterium optymalności.

Założmy, że nasz algorytm wybrał porządek topologiczny $T = t_1,t_2, \dots, t_n$, podczas gdy istnieje porządek topologiczny optymalny leksykograficznie $L = l_1, l_2, \dots, l_n$, taki że $T \not = L$. Rozważmy pierwszą, najbardziej znaczącą dla porządku leksykograficznego, pozycję $i$ na której rozwiązania się różnią. Zauważmy, że:

Wszystkie elementy $l_1, \dots, l_{i-1}$ muszą:
\begin{itemize}
\item zawierać wszystkie wierzchołki będące początkami krawędzi do $l_i$;
\item wszystkie wierzchołki nie posiadające z $l_i$ żadnej krawędzi, ale będące optymalnym wyborem względem kryterium leksykograficznego;
\item po usunięciu z grafu $G$ wszystkich krawędzi wychodzących z wierzchołków $l_1,l_2, \dots, l_{i-1}$, stopień wchodzący wierzchołka $l_i$ wynosi $0$
\end{itemize}

Zatem $l_1,l_2, \dots, l_{i-1} = t_1, t_2, \dots, t_{i-1}$, oraz $t_i > l_i$. 
Oznacza to, że w naszym rozwiązaniu $T$, po usunięciu wszystkich krawędzi prowadzących z wierzchołków $t_1,t_2, \dots, t_{i-1}$, istniał wierzchołek o stopniu wchodzącym $0$, którego indeks był mniejszy niż $t_i$. A to daje nam sprzeczność, ponieważ na tym etapie wybieramy do rozwiązania wierzchołek o najmniejszym indeksie.

\section{}%4
\fbox{\parbox[t]{\textwidth}{
Niech $u$ i $v$ będą dwoma wierzchołkami w grafie nieskierowanym $G = (V,E,c)$, gdzie $c: E \to R_{+}$ jest funkcją wagową. 
Mówimy, że droga z $u = u_1,u_2, \dots,u_{k−1},u_k = v$ z $u$ do $v$ jest sensowna, jeżeli dla każdego $i = 2, \dots , k$ istnieje droga z $u_i$ do $v$ krótsza od każdej drogi z $u_{i−1}$ do $v$ 
(przez długość drogi rozumiemy sumę wag jej krawędzi).
Ułóż algorytm, który dla danego $G$ oraz wierzchołków $u$ i $v$ wyznaczy liczbę sensownych dróg z $u$ do $v$.
}}\\

Wywołujemy algorytm \textbf{Dijkstry} dla wierzchołka $v$. Algorytm działa w czasie $O(|E| + |V| log |V|)$.
Otrzymujemy tym samym informację o długości najkrótszych dróg prowadzących z każdego wierzchołka grafu do wierzchołka $v$. Niech $S: V \to R_+$ będzie funkcją zwracającą najkrótszą drogę z wierzchołka $V$ do $v$. Wtedy:
Niech $T$ będzie listą wierzchołków grafu, posortowaną niemalejąco według długości najkrótszej drogi $S$. Niech $x$ w tej tablicy będą początkowo $0$ i oznaczają liczbę sensownych dróg.

$T = [ (d_1, S(d_1), x_1), (d_2, S(d_2), x_2),\dots, (d_{n-1}, S(d_{n-1}), x_{n-1})]$

Teraz, przechodząc listę $T$ od lewej strony do prawej:
\begin{itemize}
\item jeżeli nasz $x_i$ jest incydentalny z wierzchołkiem $v$, to dodajemy do $x_i$ jedynkę. Oznacza ona sensowną drogę, będącą najkrótszą drogą prowadzącą z $d_i$ do $v$.
\item do naszego $x_i$ dodajemy $x_j$ wszystkich wierzchołków leżących na lewo ($j < i$), z którymi nasz wierzchołek jest incydentalny. 

Drogi takie oznaczają wszystkie drogi spełniające warunek: dla każdego $i = 2, \dots , k$ droga z $u_i$ do $v$ jest krótsza od każdej drogi z $u_{i−1}$ do $v$ (czyli w szczególności od drogi najkrótszej).

\end{itemize}
Po przejściu w ten sposób całej tablicy, znamy ilość sensownych dróg z każdego wierzchołka do wierzchołka $v$.

Djikstra kosztuje $O(|E| + |V| log |V|)$, sortowanie listy $O(|V| log |V|)$, listę przeglądamy liniowo względem długości $O(|V|)$, dodawań $x'ów$ mamy, co najwyżej, $O(|E|)$. Razem $O(|E| + |V| log |V|)$.



\section{}%5
\fbox{\parbox[t]{\textwidth}{
Ułóż algorytm, który dla zadanego acyklicznego grafu skierowanego $G$ znajduje długość najdłuższej drogi w $G$. Następnie zmodyfikuj algorytm tak, by wypisywał drogę o największej długości (jeśli jest kilka takich dróg, to Twój algorytm powinien wypisać dowolną z nich).
}}
\\
\\
LengthTo - tablica $|V(G)|$ elementów początkowo równych $0$.\\
TopOrder(G) - posortowane topologicznie wierzchołki grafu $G$.\\


\begin{lstlisting}
    for each v in topOrder(G) do |v|
        for each edge (v, w) in E(G) do |w|
            if LengthTo[w] <= LengthTo[v] + weight(v, w) then
               LengthTo[w]  = LengthTo[v] + weight(v, w)
 
    return max(LengthTo[v] for v in V(G))
\end{lstlisting}

Przechodzimy topologicznie wierzchołki grafu, dzięki temu, przechodząc listę wierzchołków wiemy, że rozpatrzyliśmy już wszystkie wierzchołki które prowadzą do aktualnie przeglądanego wierzchołka. Czyli, dla wierzchołka $v$ rozpatrzyliśmy wszystkie drogi do niego prowadzące, i znamy najdłuższą.

Dla każdego wierzchołka na liście sprawdzamy, czy wierzchołek do którego możemy z niego dojść posiada dłuższą drogę. W ten sposób zachłannie wybieramy zawsze najdłuższą drogę jaką można dojść do każdego wierzchołka, a ponieważ robimy to w porządku topologicznym 

Sortowanie topologiczne działa w czasie $O(E+V)$,
więc całość działa w czasie $O(E+V+E+V) = O(E+V)$.

Aby wypisać drogę musimy tylko zapamiętać, dla których wierzchołków spełniony był warunek:

Prev - tablica początkowo równa $nil$.\\

\begin{lstlisting}
def lognest G
    for each vertex V in topOrder(G) do
        for each edge (V, W) in E(G) do
            if LengthTo[W] <= LengthTo[V] + weight(G,(V,W)) then
               LengthTo[W]  = LengthTo[V] + weight(G,(V,W))
               Prev[W] = V
 
    return max(LengthTo[V] for V in V(G))
    
def longest_path G
	x = longest(G)
	do
		print x
		x = Prev[x]
	until x.nil?
\end{lstlisting}

\subsection{szkic dowodu poprawności kryterium optymalizacyjnego}
Załóżmy że w $i$tym kroku nasz algorytm wylicza nieoptymalną długość drogi do wierzchołka $a$. Ma to znaczenie tylko w momencie, kiedy rozpatrujemy krawędzie wychodzące z $a$. Ponieważ, rozpatrzyliśmy do tego momentu wszystkie wierzchołki grafu prowadzące do $a$, sprawdziliśmy długość wszystkich możliwych dróg prowadzących do $a$ i zachłannie wybraliśmy największą. 

\section{}%6

\end{document}