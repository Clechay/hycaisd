% !TEX encoding = UTF-8 Unicode
\documentclass[svgnames]{report}
\usepackage[utf8x]{inputenc} 
\usepackage{polski}       
\usepackage{a4wide}
\usepackage{graphicx}
\usepackage{amsmath,amssymb}
\usepackage{bbm}            % sudo apt-get install texlive-fonts-recommended texlive-fonts-extra
\usepackage{amsthm}
\usepackage{algorithmic}	% sudo apt-get install texlive-science
\usepackage{listings}             % Include the listings-package
\usepackage{framed}
\usepackage{enumerate}
\usepackage{hyperref}


\makeatletter
 \renewcommand\@seccntformat[1]{\csname  the#1\endcsname.\quad}
  

\lstset{language=C}
%\DeclareUnicodeCharacter{00A0}{ }

\begin{document}
%\chapter{lista 0}
%%%%%%%%%%%%%%%%%%%%%%%%%%%%%%%%%%%%%%%%%%%%%%%%%%%%%%%%%%%%%%%%%%%%
%\chapter{Lista 1}

\section{}
\section{}
\section{}
\section{}
\[a_1 = a, \; a_k = 1, \; a_{i+1} = \lfloor \frac{a_i}{2}\rfloor\]
\[b_a = b, \; b_{i+1} = 2b_i \Rightarrow b_i = 2^ib \] 
Niech $a = \sum_{i=1}^k 2^{i-1}\bar{a}_i$, $\bar{a}_i \in \{ 0,1\}$, czyli zapis w postaci binarnej. Wtedy

\[a_n = \sum_{i=1}^n 2^{i-1}\bar{a}_{i+(k-n)}\]

Dowód:

\[\sum_{i=1, \; nieparzyste \; a_i}^k b_i = \sum_{i=1}^k\bar{a}_i2^ib = b\sum_{i=1}^k2^i\bar{a}_i = ab\]

Kryterium jednorodne - koszt każdej operacji jest jednostkowy. Zatem ile jedynek w zapisie binarnym liczby $b$, taką mamy złożoność $\Rightarrow O(\lceil(log_2b)\rceil)$ \\
Kryterium lagarytmiczne - koszt operacji zależy od odługości operandów. Dodawań mamy tyle samo $\Rightarrow O(\lceil log_2b\rceil \cdot \lceil log_2 ab\rceil)$



\section{}

Niech $f_n = A_kf_{n-k} + A_{k-1}f_{n-k+1} + \ldots A_1f_{n-1}$. Jeżeli $f_n$ zależy od $k$ poprzednich lementów to tworzymy następującą macierz $k\times k$(x-kolumny, y-wiersze, indeksowanie od 0)

\[M_{x,y} = 1, \; dla \;x=y+1\]
\[M_{x,k} = A_{n- k + x}\]
Reszta elementów to 0. Przykład dla $k=3$
\[ 
 \left[
 	\begin{array}{ccc}
 	0		&	1		&	0\\
 	0		&	0		&	1\\
 	A_{n-3}	&	A_{n-2}	&	A_{n-1}
 	\end{array}
 \right]
\] 

Żeby policzyć $f_n$ tworzymy wektor $F = (f_{n-k} \ldots f_{n-1})$ i obliczamy $M\cdot F^T$.\newline


Tworzymy macierz rozmiaru $(k+m) \times (k+m)$, $k$ - ilość poprzednich wyrazów ciągu, $m$ - stopień wielomianu.
Dzielimy ją na cztery prostokąty. Lewy górny taki jak w poprzedniej części. Prawy górny wypełniamy zerami, oprócz ostatniego wiersza, który wypełniamy współczynnikami wielomianu. Lewy dolny wypełniamy zerami. Żeby wypełnić prawy dolny zastanówmy się najpierw jak uzyskać $(n+1)^i$. Oczywiście z dwumianu newtona. prawy dolny prostokąt będzie miał postać

\[
 \left[
 	\begin{array}{cccc}
 	{m \choose m}	&	\ldots				&	\ldots	& 	{m \choose 0}\\
 	0 				&	{m-1\choose m-1}	&	\ldots	&	{m-1 \choose 0}\\
 	\ldots			& 	\ldots 				& 	\ldots	& 	\ldots\\
 	0				&	0					&	\ldots	&	1
 	\end{array}
 \right]
\]


Znowu tworzymy sobie wektor $F = (f_{n-k} \ldots f_{n-1}, n^m, n^{m-1}, \dots, 1)$ i obliczamy $M\cdot F^T$.

\section{}


\section{}

Mamy daną listę $L$. Dzielimy ją na podlisty długości $\sqrt(n)$. Tworzymy dodatkową listę $K$ o długości $\sqrt(n)$, zawierającą wskaźniki na pierwsze elementy utworzonych wcześniej podlist. Przy wstawianiu elementu przeglądamy najpierw Listę $K$, a następnie listę $L$ od miejsca, na które wskazywał wskaźnik z $K$. Maksymalnie przejrzymy $2\sqrt(n)$ elementów. Po wstawieniu elementu musimy ouaktualnić listę wskaźników $K$, czego koszt to znowu $\sqrt(n)$

\section{}

Wejście: Skierowany acykliczny graf.\\
Wyjście: Długość najdłuższej ścieżki.\\
LengthTo - tablica $|V(G)|$ elementów początkowo równych 0.\\
TopOrder(G) - posortowane topologicznie wierzchołki.\\


\begin{lstlisting}
    for each vertex V in topOrder(G) do
        for each edge (V, W) in E(G) do
            if LengthTo[W] <= LengthTo[V] + weight(G,(V,W)) then
               LengthTo[W]  = LengthTo[V] + weight(G,(V,W))
 
    return max(LengthTo[V] for V in V(G))
\end{lstlisting}

Sortowanie topologiczne działa w czasie $O(E+V)$, więc całość działa w czasie $O(E+V+E+V) = O(E+V)$.
Żeby wypisać drogę musimy tylko zapamiętywać, dla których wierzchołków spełniony był IF.



\chapter{Lista 1.}

\section{}%1
\framebox[\textwidth]{Napisz rekurencyjne funkcje, które dla danego drzewa binarnego $T$ obliczają:}
\begin{itemize}
\item{liczbę wierzchołków $T$}
\begin{lstlisting}
def nodes tree
	return 0 if tree.nil?
	return 1 if tree.leaf?
	return 1 + nodes(tree.left) + nodes(tree.right)
\end{lstlisting}

\item{maksymalną odległość między wierzchołkami w $T$}
Jest to \textbf{średnica} drzewa.
Zauważmy, że:
\begin{itemize}
\item Puste drzewo ma średnicę $0$.
\item Jeśli drzewo jest niepuste, to przez $t_1$ i $t_2$ oznaczmy dwa poddrzewa zakorzenione w lewym i prawym synie korzenia. Odpowiednio przez $d_1$ i $d_2$ oznaczmy średnice tych poddrzew, a przez $h_1$ i $h_2$ ich wysokości. Wówczas średnica całego drzewa wynosi $max(d_1, d_2, h_1+h_2+2)$.
\end{itemize}
\begin{lstlisting}
def diameter(node) 
	return (0,0) if node.nil? 
	(lheight, ldiameter) = diameter(node.left)
	(rheight, rdiameter) = diameter(node.right)
	
	height = max(lheight, rheight) + 1
	diameter = max(lheight + rheight + 2, ldiameter, rdiameter)

	return [height, diameter]
\end{lstlisting}
\end{itemize}
\section{}%2
\fbox{\parbox[t]{\textwidth}{Dla kopca minimaksowego. Przyjmij, że elementy pamiętane są w jednej tablicy (określ w jakiej kolejności). Napisz w pseudokodzie procedury: }}
\begin{itemize}
\item{przywracania porządku}
\item{usuwania minimum}
\item{usuwania maksimum}

\end{itemize}
\section{}%3
\fbox{\parbox[t]{\textwidth}{
Porządkiem topologicznym wierzchołków acyklicznego digrafu $G = (V, E)$ nazywamy taki liniowy porządek jego wierzchołków, w którym początek każdej krawędzi występuje przed jej końcem. Jeżeli wierzchołki z $V$ utożsamimy z początkowymi liczbami naturalnymi to każdy ich przodek liniowy można opisać permutacją liczb $1,2,3,...,|V|$; w szczególności pozwala to na porównywanie leksykograficzne porządków.
Ułóż algorytm, który dla danego digrafu znajduje pierwszy leksykograficznie porządek.}}

$Q$ - Kolejka priorytetowa z wierzchołkami o stopniu wchodzącym równym $0$.


\begin{verbatim}
dopóki Q jest niepusta rób
    usuń wierzchołek n z przodu kolejki Q
    wypisz n
    dla każdego wierzchołka m o krawędzi e od n do m rób
        usuń krawędź e z grafu
        jeżeli do m nie prowadzi żadna krawędź to
            wstaw m do Q
jeżeli graf ma wierzchołki to
    wypisz komunikat o błędzie (graf zawiera cykl)
\end{verbatim}    

Złożoność wynosi $O(|E| + |V| log |V|)$.
\section{}%4
\fbox{\parbox[t]{\textwidth}{
Niech $u$ i $v$ będą dwoma wierzchołkami w grafie nieskierowanym $G = (V,E,c)$, gdzie $c: E \to R_{+}$ jest funkcją wagową. 
Mówimy, że droga z $u = u_1,u_2, \dots,u_{k−1},u_k = v$ z $u$ do $v$ jest sensowna, jeżeli dla każdego $i = 2, \dots , k$ istnieje droga z $u_i$ do $v$ krótsza od każdej drogi z $u_{i−1}$ do $v$ 
(przez długość drogi rozumiemy sumę wag jej krawędzi).
Ułóż algorytm, który dla danego $G$ oraz wierzchołków $u$ i $v$ wyznaczy liczbę sensownych dróg z $u$ do $v$.
}}\\
Djikstra i lecimy wyniki na lewo. TBC
\section{}%5
Wejście: Skierowany acykliczny graf.\\
\\
LengthTo - tablica $|V(G)|$ elementów początkowo równych 0\\
TopOrder(G) - posortowane topologicznie wierzchołki.\\


\begin{lstlisting}
    for each vertex V in topOrder(G) do
        for each edge (V, W) in E(G) do
            if LengthTo[W] <= LengthTo[V] + weight(G,(V,W)) then
               LengthTo[W]  = LengthTo[V] + weight(G,(V,W))
 
    return max(LengthTo[V] for V in V(G))
\end{lstlisting}

Sortowanie topologiczne działa w czasie $O(E+V)$,
więc całość działa w czasie $O(E+V+E+V) = O(E+V)$.
a żeby wypisać drogę musimy tylko zapamiętać, dla których wierzchołków spełniony był IF.

\section{}%6
\section{Druga wersja zadania 1.5.}%7

 w wersji O(n+m)
Q - kolejka wierzchołków o indeg = 0
P - kolejka wierzchołków posortowanych topologicznie
T - wyzerowana tablica o dlugosci |V|
\begin{lstlisting}
while !Q.empty:
v = Q.pop();
   P.;push_back(v);
      foreach e=(v,u) in G:
            delete(e);
                  if(in_deg(u) == 0):
                            Q.push(u);
                            
                            foreach v in P:
                               foreach (v,u) in G:
                                     if T[u] < T[v] + w(v,u):
                                                T[u] = T[v] + w(v,u);
                                                
                                                return max(T[v] for v in G)
\end{lstlisting}
żeby była rekonstrukcja ścieżki to tak:
wierzchołek końcowy będziemy mieli po wyznaczeniu z tego maxa
więc potrzebna nam tylko tabela poprzedników PREV[v] dla v w G
i tą tablicę wystarczy wypełniać przy podstawianiu nowej wartości pod T[u]


\end{document}