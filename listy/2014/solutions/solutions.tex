% !TEX encoding = UTF-8 Unicode
\documentclass[svgnames]{report}
\usepackage[utf8x]{inputenc} 
\usepackage{polski}       
\usepackage{a4wide}
\usepackage{graphicx}
\usepackage{amsmath,amssymb}
\usepackage{bbm}            % sudo apt-get install texlive-fonts-recommended texlive-fonts-extra
\usepackage{amsthm}
\usepackage{algorithmic}	% sudo apt-get install texlive-science
\usepackage{listings}             % Include the listings-package
\usepackage{framed}
\usepackage{enumerate}
\usepackage{hyperref}


\makeatletter
 \renewcommand\@seccntformat[1]{\csname  the#1\endcsname.\quad}
  

\lstset{language=C}
%\DeclareUnicodeCharacter{00A0}{ }

\begin{document}
%\chapter{lista 0}
%%%%%%%%%%%%%%%%%%%%%%%%%%%%%%%%%%%%%%%%%%%%%%%%%%%%%%%%%%%%%%%%%%%%
%\chapter{Lista 1}

\section{}
\section{}
\section{}
\section{}
\[a_1 = a, \; a_k = 1, \; a_{i+1} = \lfloor \frac{a_i}{2}\rfloor\]
\[b_a = b, \; b_{i+1} = 2b_i \Rightarrow b_i = 2^ib \] 
Niech $a = \sum_{i=1}^k 2^{i-1}\bar{a}_i$, $\bar{a}_i \in \{ 0,1\}$, czyli zapis w postaci binarnej. Wtedy

\[a_n = \sum_{i=1}^n 2^{i-1}\bar{a}_{i+(k-n)}\]

Dowód:

\[\sum_{i=1, \; nieparzyste \; a_i}^k b_i = \sum_{i=1}^k\bar{a}_i2^ib = b\sum_{i=1}^k2^i\bar{a}_i = ab\]

Kryterium jednorodne - koszt każdej operacji jest jednostkowy. Zatem ile jedynek w zapisie binarnym liczby $b$, taką mamy złożoność $\Rightarrow O(\lceil(log_2b)\rceil)$ \\
Kryterium lagarytmiczne - koszt operacji zależy od odługości operandów. Dodawań mamy tyle samo $\Rightarrow O(\lceil log_2b\rceil \cdot \lceil log_2 ab\rceil)$



\section{}

Niech $f_n = A_kf_{n-k} + A_{k-1}f_{n-k+1} + \ldots A_1f_{n-1}$. Jeżeli $f_n$ zależy od $k$ poprzednich lementów to tworzymy następującą macierz $k\times k$(x-kolumny, y-wiersze, indeksowanie od 0)

\[M_{x,y} = 1, \; dla \;x=y+1\]
\[M_{x,k} = A_{n- k + x}\]
Reszta elementów to 0. Przykład dla $k=3$
\[ 
 \left[
 	\begin{array}{ccc}
 	0		&	1		&	0\\
 	0		&	0		&	1\\
 	A_{n-3}	&	A_{n-2}	&	A_{n-1}
 	\end{array}
 \right]
\] 

Żeby policzyć $f_n$ tworzymy wektor $F = (f_{n-k} \ldots f_{n-1})$ i obliczamy $M\cdot F^T$.\newline


Tworzymy macierz rozmiaru $(k+m) \times (k+m)$, $k$ - ilość poprzednich wyrazów ciągu, $m$ - stopień wielomianu.
Dzielimy ją na cztery prostokąty. Lewy górny taki jak w poprzedniej części. Prawy górny wypełniamy zerami, oprócz ostatniego wiersza, który wypełniamy współczynnikami wielomianu. Lewy dolny wypełniamy zerami. Żeby wypełnić prawy dolny zastanówmy się najpierw jak uzyskać $(n+1)^i$. Oczywiście z dwumianu newtona. prawy dolny prostokąt będzie miał postać

\[
 \left[
 	\begin{array}{cccc}
 	{m \choose m}	&	\ldots				&	\ldots	& 	{m \choose 0}\\
 	0 				&	{m-1\choose m-1}	&	\ldots	&	{m-1 \choose 0}\\
 	\ldots			& 	\ldots 				& 	\ldots	& 	\ldots\\
 	0				&	0					&	\ldots	&	1
 	\end{array}
 \right]
\]


Znowu tworzymy sobie wektor $F = (f_{n-k} \ldots f_{n-1}, n^m, n^{m-1}, \dots, 1)$ i obliczamy $M\cdot F^T$.

\section{}


\section{}

Mamy daną listę $L$. Dzielimy ją na podlisty długości $\sqrt(n)$. Tworzymy dodatkową listę $K$ o długości $\sqrt(n)$, zawierającą wskaźniki na pierwsze elementy utworzonych wcześniej podlist. Przy wstawianiu elementu przeglądamy najpierw Listę $K$, a następnie listę $L$ od miejsca, na które wskazywał wskaźnik z $K$. Maksymalnie przejrzymy $2\sqrt(n)$ elementów. Po wstawieniu elementu musimy ouaktualnić listę wskaźników $K$, czego koszt to znowu $\sqrt(n)$

\section{}

Wejście: Skierowany acykliczny graf.\\
Wyjście: Długość najdłuższej ścieżki.\\
LengthTo - tablica $|V(G)|$ elementów początkowo równych 0.\\
TopOrder(G) - posortowane topologicznie wierzchołki.\\


\begin{lstlisting}
    for each vertex V in topOrder(G) do
        for each edge (V, W) in E(G) do
            if LengthTo[W] <= LengthTo[V] + weight(G,(V,W)) then
               LengthTo[W]  = LengthTo[V] + weight(G,(V,W))
 
    return max(LengthTo[V] for V in V(G))
\end{lstlisting}

Sortowanie topologiczne działa w czasie $O(E+V)$, więc całość działa w czasie $O(E+V+E+V) = O(E+V)$.
Żeby wypisać drogę musimy tylko zapamiętywać, dla których wierzchołków spełniony był IF.



%\chapter{Lista 1}

\section{}
\section{}
\section{}
\section{}
\[a_1 = a, \; a_k = 1, \; a_{i+1} = \lfloor \frac{a_i}{2}\rfloor\]
\[b_a = b, \; b_{i+1} = 2b_i \Rightarrow b_i = 2^ib \] 
Niech $a = \sum_{i=1}^k 2^{i-1}\bar{a}_i$, $\bar{a}_i \in \{ 0,1\}$, czyli zapis w postaci binarnej. Wtedy

\[a_n = \sum_{i=1}^n 2^{i-1}\bar{a}_{i+(k-n)}\]

Dowód:

\[\sum_{i=1, \; nieparzyste \; a_i}^k b_i = \sum_{i=1}^k\bar{a}_i2^ib = b\sum_{i=1}^k2^i\bar{a}_i = ab\]

Kryterium jednorodne - koszt każdej operacji jest jednostkowy. Zatem ile jedynek w zapisie binarnym liczby $b$, taką mamy złożoność $\Rightarrow O(\lceil(log_2b)\rceil)$ \\
Kryterium lagarytmiczne - koszt operacji zależy od odługości operandów. Dodawań mamy tyle samo $\Rightarrow O(\lceil log_2b\rceil \cdot \lceil log_2 ab\rceil)$



\section{}

Niech $f_n = A_kf_{n-k} + A_{k-1}f_{n-k+1} + \ldots A_1f_{n-1}$. Jeżeli $f_n$ zależy od $k$ poprzednich lementów to tworzymy następującą macierz $k\times k$(x-kolumny, y-wiersze, indeksowanie od 0)

\[M_{x,y} = 1, \; dla \;x=y+1\]
\[M_{x,k} = A_{n- k + x}\]
Reszta elementów to 0. Przykład dla $k=3$
\[ 
 \left[
 	\begin{array}{ccc}
 	0		&	1		&	0\\
 	0		&	0		&	1\\
 	A_{n-3}	&	A_{n-2}	&	A_{n-1}
 	\end{array}
 \right]
\] 

Żeby policzyć $f_n$ tworzymy wektor $F = (f_{n-k} \ldots f_{n-1})$ i obliczamy $M\cdot F^T$.\newline


Tworzymy macierz rozmiaru $(k+m) \times (k+m)$, $k$ - ilość poprzednich wyrazów ciągu, $m$ - stopień wielomianu.
Dzielimy ją na cztery prostokąty. Lewy górny taki jak w poprzedniej części. Prawy górny wypełniamy zerami, oprócz ostatniego wiersza, który wypełniamy współczynnikami wielomianu. Lewy dolny wypełniamy zerami. Żeby wypełnić prawy dolny zastanówmy się najpierw jak uzyskać $(n+1)^i$. Oczywiście z dwumianu newtona. prawy dolny prostokąt będzie miał postać

\[
 \left[
 	\begin{array}{cccc}
 	{m \choose m}	&	\ldots				&	\ldots	& 	{m \choose 0}\\
 	0 				&	{m-1\choose m-1}	&	\ldots	&	{m-1 \choose 0}\\
 	\ldots			& 	\ldots 				& 	\ldots	& 	\ldots\\
 	0				&	0					&	\ldots	&	1
 	\end{array}
 \right]
\]


Znowu tworzymy sobie wektor $F = (f_{n-k} \ldots f_{n-1}, n^m, n^{m-1}, \dots, 1)$ i obliczamy $M\cdot F^T$.

\section{}


\section{}

Mamy daną listę $L$. Dzielimy ją na podlisty długości $\sqrt(n)$. Tworzymy dodatkową listę $K$ o długości $\sqrt(n)$, zawierającą wskaźniki na pierwsze elementy utworzonych wcześniej podlist. Przy wstawianiu elementu przeglądamy najpierw Listę $K$, a następnie listę $L$ od miejsca, na które wskazywał wskaźnik z $K$. Maksymalnie przejrzymy $2\sqrt(n)$ elementów. Po wstawieniu elementu musimy ouaktualnić listę wskaźników $K$, czego koszt to znowu $\sqrt(n)$

\section{}

Wejście: Skierowany acykliczny graf.\\
Wyjście: Długość najdłuższej ścieżki.\\
LengthTo - tablica $|V(G)|$ elementów początkowo równych 0.\\
TopOrder(G) - posortowane topologicznie wierzchołki.\\


\begin{lstlisting}
    for each vertex V in topOrder(G) do
        for each edge (V, W) in E(G) do
            if LengthTo[W] <= LengthTo[V] + weight(G,(V,W)) then
               LengthTo[W]  = LengthTo[V] + weight(G,(V,W))
 
    return max(LengthTo[V] for V in V(G))
\end{lstlisting}

Sortowanie topologiczne działa w czasie $O(E+V)$, więc całość działa w czasie $O(E+V+E+V) = O(E+V)$.
Żeby wypisać drogę musimy tylko zapamiętywać, dla których wierzchołków spełniony był IF.



\chapter{}
\chapter{Lista 2.}
\section{0pkt}
\begin{framed}
Przeczytaj notatkę o algorytmach zachłannych...
\end{framed}
%%%%%%%%%%%%%%%
\section{1pkt}
\begin{framed}
Danych jest $n$ odcinków $I_j = <p_j,k_j>$, leżących na osi $OX$, $j = 1,\dots,n$. Uałóż algorytm
znajdujący zbiór $S \subseteq \{ I_1,\dots,I_n \}$, nieprzecinających się odcinków, o największej mocy.
\end{framed}

Sortujemy odcinki rosnąco wg. końców $k_j$. Wybieramy pierwszy, gdyż najwcześniej się zakończy. Następnie pierwszy kolejny, który się zmieści po poprzednio wybranym.\\

\noindent Dowód: Niech nasz algorytm daje ciąg $S = \{s_1, s_2, \dots, s_n\}$. Załóżmy, że istnieje lepszy, optymalny ciąg $O = \{o_1,o_2, \dots, o_n\}$. 
Weźmy pierwszą pozycję $i$, na której rozwiązania się różnią, czyli taką, że: $s_i \ne o_i$, oraz $j < i \Rightarrow u_j = o_j$. Rozważmy dwa przypadki:
\begin{itemize}
\item Jeśli kończy się na tej samej pozycji to nie ma znaczenia, który z nich został wybrany. Zatem rozwiązanie $\{o_1,\dots,o_i\}$ nie jest lepsze od $\{s_1,\dots,s_i\}$. (sprzeczność)
\item Jeżeli odcinek $o_i$ kończy się wcześniej niż $u_i$ to zostałby on wybrany przez nasz algorytm zgodnie z lokalny, zachłannym kryterium. (sprzeczność)

\end{itemize}
Rozumowanie powtarzamy indukcyjnie dla kolejnych pozycji na których rozwiązania się różnią. Algorytm zachłanny daje więc rozwiązanie optymalne.


%%%%%%%%%%%%%%%
\section{1pkt}
\begin{framed}
Rozważ następującą wersję problemu wydawania reszty: dla danych liczb naturalnych
$a, b$ $(a \leq b)$ chcemy przedstawić ułamek $\frac{a}{b}$ jako sumę różnych ułamków o licznikach równych
$1$. Udowodnij, że algorytm zachłanny zawsze daje rozwiązanie. Czy zawsze jest to rozwiązanie optymalne (tj. o najmniejszej liczbie składników)?
\end{framed}

Wytwarzanie ułamka egipskiego mniejszego od $\frac{x}{y}$ o najwiekszym mianowniku: 
\[\frac{x}{y} \rightarrow \frac{1}{\lceil \frac{y}{x} \rceil}\]

\noindent Licznik wyrażenia $\frac{x}{y} - \frac{1}{\lceil \frac{y}{x} \rceil} = \frac{x \lceil \frac{y}{x} \rceil - y}{x\lceil \frac{y}{x} \rceil}$ będzie malał, lecz zawsze będzie $\ge 0$:\\

\noindent Sprawdźmy, czy faktycznie $ 0 \leq x \lceil \frac{y}{x} \rceil - y < x$

\[x \lceil \frac{y}{x} \rceil - y < x \Leftrightarrow x \lceil \frac{y}{x} \rceil - x < y \Leftrightarrow \lceil \frac{y}{x} \rceil - 1 < \frac{y}{x} \textrm{.}\]

\noindent Co jest oczywiście prawdą. Dodatkowo: $x \lceil \frac{y}{x} \rceil - y \ge 0$. Więc ułamek zawsze się zmiejsza, i zatrzyma się na $0$.

\noindent Kontrprzykład na optymalność. Nasz da jakieś gówno, optymalny da:

\[\frac{5}{121}=\frac{1}{33}+\frac{1}{121}+\frac{1}{363}\]


%%%%%%%%%%%%%%%
\section{2pkt}
\begin{framed}
Udowodnij poprawność algorytmu Boruvki Solina.
\end{framed}


\subsection{Lemat 1}
\noindent W każdym momencie działania algorytmu, oraz po jego zakończeniu w $E'$ nie będzie cyklu.\\

\noindent Dowód: Załóżmy nie wprost, że podczas działania algorytmu w którymś etapie pojawiła się spójna składowa, w której jest cykl. Oznaczmy ją jako $S$. Rozważmy następujące sytuacje:\\

\begin{itemize}
\item $S$ powstała przez połączenie dwóch superwierzchołków $v_1$ i $v_2$. Oznacza to, że do zbioru $E'$ zostały dołączone krawędzie $e_i$ i $e_j$. Ponieważ $e_i$ została dołączona jako najlżejsza krawędź incydentna do $v_1$ więc $C(e_i) < C(e_j)$. Ale skoro $e_j$ została dołączona jako najlżejsza krawędź incydentna do $v_2$ to musi zachodzić  (Pamiętajmy, że w grafie nie ma krawędzi o takiej samej wadze) - sprzeczność.

\item $S$ powstała przez połączenie się trzech lub więcej superwierzchołków. Podzielmy powstały cykl $C$ na następujące części: niech $v_1,.../v_l$ będą kolejnymi superwierzchołkami należącymi do $C$ a $e_1,...,e_l$ będą kolejnymi krawędziami należącymi do $C$, które zostały dodane w zakończonym właśnie etapie algorytmu. W $C$ krawędzie $e_i$ oraz superwierzchołki $v_i$ występują na przemian. Z zasady działania algorytmu możemy stwierdzić, że aby powstał taki cykl, musi zachodzić $C(e_1) < C(e_2) ... < C(e_l) < C(e_1)$ - Sprzeczność.
\end{itemize}

\subsection{Lemat 2}

\noindent W każdym etapie działania algorytmu otrzymujemy dla każdego superwierzchołka minimalne drzewo rozpinające.\\

\noindent Dowód:

\begin{itemize}
\item Gdy zostanie zakończony etap 1:

Załóżmy, że istnieje taki superwierzchołek $v_i$, który nie jest minimalnym drzewem rozpinającym poddrzewa złożonego z wierzchołków należących do $v_i$. Weźmy więc takie minimalne drzewo rozpinające $T$. Istnieje krawędź $e_i$ taka, że $e_i \in E(v_i)$ oraz $e_i \not\in E(T)$ . Dodajmy $e_i$. W $T$ powstał cykl. Ponieważ $e_i$ jest incydenta do pewnego wierzchołka z tego cyklu, istnieje więc inna krawędź $e^\prime_i$ incydentna do tego samego wierzchołka. Jednak z tego, że $e_i \in E(v_i)$ wynika, że $C(e_i) < C(e^\prime_i)$. Jeśli usuniemy krawędź $e^\prime_i$ z $T$ otrzymamy mniejsze drzewo rozpinające, co jest sprzeczne z założeniem o minimalności $T$.

\item Gdy zostanie zakończony etap 2:

Z poprawności etapu 1 wiemy, że istnieje takie wywołanie etapu 2, w którym każdy z superwierzchołków jest minimalnym drzewem rozpinającym. Jest to choćby pierwsze wywołanie. Załóżmy zatem, że dla pewnego wywołania tego etapu otrzymano superwierzchołki będące minimalnymi drzewami rozpinającymi, jednak scaliło przynajmniej dwa z nich w taki sposób, że dało się otrzymać mniejsze drzewo rozpinające. Niech etap k-ty będzie pierwszym takim etapem, w którym coś się popsuło. Niech $E^\prime_1$ będzie zbiorem krawędzi przed wywołaniem etapu k, a $E^\prime_2$ będzie zbiorem krawędzi po jego wywołaniu. Niech $T$ będzie minimalnym drzewem rozpinającym takim, że $V(T) = V(v_i)$, ale że $E(T) \neq E(v_i)$ . Istnieje więc krawędź $e_i \in E(v_i)$ oraz $e_i \not\in E(T)$.\\

Fakt: Krawędź  dodana podczas k-tego wywołania. (Nie może należeć do $E^\prime_1$ gdyż inaczej superwierzchołek do którego by należała nie byłby minimalnym drzewem rozpinającym, co jest sprzeczne z dowodem dla pierwszego etapu i założeniem, że wywołanie k-te jest najmniejszym wywołaniem, które zwróciło nieoptymalne drzewa)
Dodajmy krawędź $e_i$ do $E(T)$. W $T$ powstał cykl. Ponieważ $e_i$ jest najmniejszą krawędzią incydentną do pewnego superwierzchołka z tego cyklu, istnieje więc inna krawędź incydenta do tego samego superwierzchołka. Jednak jej waga jest większa niż waga krawędzi $e_i$, zatem zastąpienie jej krawędzią $e_i$ da nam mniejsze drzewo rozpinające co jest sprzeczne z założeniem o optymalności T.
\end{itemize}



%%%%%%%%%%%%%%%
\section{2pkt}
\begin{framed}
Ułóż algorytm, który dla danego spójnego grafu $G$ oraz krawędzi $e$ sprawdza w czasie $O(n + m)$, czy krawędź $e$ należy do jakiegoś minimalnego drzewa spinającego grafu $G$. Możesz założyć, że wszystkie wagi krawędzi są różne.
\end{framed}

%%%%%%%%%%%%%%%
\section{}
\begin{framed}
System złożony z dwóch maszyn $A$ i $B$ wykonuje $n$ zadań. Każde z zadań wykonywane jest na obydwu maszynach, przy czym wykonanie zadania na maszynie $B$ można rozpocząć dopiero po zakończeniu wykonywania go na maszynie $A$. Dla każdego zadania określone są dwie liczby naturalne $a_i$ i $b_i$ określające czas wykonania i-tego zadania na maszynie $A$ oraz $B$ (odpowiednio). Ułóż algorytm ustawiający zadania w kolejności minimalizującej czas zakończenia wykonania ostatniego zadania praz maszynę $B$.
\end{framed}


\end{document}