% !TEX encoding = UTF-8 Unicode
\documentclass[svgnames]{report}
\usepackage[utf8]{inputenc} 
\usepackage{polski}       
\usepackage{a4wide}
\usepackage{graphicx}
\usepackage{amsmath,amssymb}
\usepackage{bbm}            % sudo apt-get install texlive-fonts-recommended texlive-fonts-extra
\usepackage{amsthm}
\usepackage{algorithmic}	% sudo apt-get install texlive-science
\usepackage{listings}             % Include the listings-package
\usepackage{framed}
\usepackage{enumerate}
\usepackage{hyperref}


\makeatletter
 \renewcommand\@seccntformat[1]{\csname  the#1\endcsname.\quad}
  

\lstset{language=C}


\begin{document}

%%%%%%%%%%%%%%%%%%%%%%%%%%%%%%%%%%%%%%%%%%%%%%%%%%%%%%%%%%%%%%%%%%%%
\chapter{Lista 1}

\section{}
\section{}
\section{}
\section{}
\[a_1 = a, \; a_k = 1, \; a_{i+1} = \lfloor \frac{a_i}{2}\rfloor\]
\[b_a = b, \; b_{i+1} = 2b_i \Rightarrow b_i = 2^ib \] 
Niech $a = \sum_{i=1}^k 2^{i-1}\bar{a}_i$, $\bar{a}_i \in \{ 0,1\}$, czyli zapis w postaci binarnej. Wtedy

\[a_n = \sum_{i=1}^n 2^{i-1}\bar{a}_{i+(k-n)}\]

Dowód:

\[\sum_{i=1, \; nieparzyste \; a_i}^k b_i = \sum_{i=1}^k\bar{a}_i2^ib = b\sum_{i=1}^k2^i\bar{a}_i = ab\]

Kryterium jednorodne - koszt każdej operacji jest jednostkowy. Zatem ile jedynek w zapisie binarnym liczby $b$, taką mamy złożoność $\Rightarrow O(\lceil(log_2b)\rceil)$ \\
Kryterium lagarytmiczne - koszt operacji zależy od odługości operandów. Dodawań mamy tyle samo $\Rightarrow O(\lceil log_2b\rceil \cdot \lceil log_2 ab\rceil)$



\section{}

Niech $f_n = A_kf_{n-k} + A_{k-1}f_{n-k+1} + \ldots A_1f_{n-1}$. Jeżeli $f_n$ zależy od $k$ poprzednich lementów to tworzymy następującą macierz $k\times k$(x-kolumny, y-wiersze, indeksowanie od 0)

\[M_{x,y} = 1, \; dla \;x=y+1\]
\[M_{x,k} = A_{n- k + x}\]
Reszta elementów to 0. Przykład dla $k=3$
\[ 
 \left[
 	\begin{array}{ccc}
 	0		&	1		&	0\\
 	0		&	0		&	1\\
 	A_{n-3}	&	A_{n-2}	&	A_{n-1}
 	\end{array}
 \right]
\] 

Żeby policzyć $f_n$ tworzymy wektor $F = (f_{n-k} \ldots f_{n-1})$ i obliczamy $M\cdot F^T$.\newline


Tworzymy macierz rozmiaru $(k+m) \times (k+m)$, $k$ - ilość poprzednich wyrazów ciągu, $m$ - stopień wielomianu.
Dzielimy ją na cztery prostokąty. Lewy górny taki jak w poprzedniej części. Prawy górny wypełniamy zerami, oprócz ostatniego wiersza, który wypełniamy współczynnikami wielomianu. Lewy dolny wypełniamy zerami. Żeby wypełnić prawy dolny zastanówmy się najpierw jak uzyskać $(n+1)^i$. Oczywiście z dwumianu newtona. prawy dolny prostokąt będzie miał postać

\[
 \left[
 	\begin{array}{cccc}
 	{m \choose m}	&	\ldots				&	\ldots	& 	{m \choose 0}\\
 	0 				&	{m-1\choose m-1}	&	\ldots	&	{m-1 \choose 0}\\
 	\ldots			& 	\ldots 				& 	\ldots	& 	\ldots\\
 	0				&	0					&	\ldots	&	1
 	\end{array}
 \right]
\]


Znowu tworzymy sobie wektor $F = (f_{n-k} \ldots f_{n-1}, n^m, n^{m-1}, \dots, 1)$ i obliczamy $M\cdot F^T$.

\section{}


\section{}

Mamy daną listę $L$. Dzielimy ją na podlisty długości $\sqrt(n)$. Tworzymy dodatkową listę $K$ o długości $\sqrt(n)$, zawierającą wskaźniki na pierwsze elementy utworzonych wcześniej podlist. Przy wstawianiu elementu przeglądamy najpierw Listę $K$, a następnie listę $L$ od miejsca, na które wskazywał wskaźnik z $K$. Maksymalnie przejrzymy $2\sqrt(n)$ elementów. Po wstawieniu elementu musimy ouaktualnić listę wskaźników $K$, czego koszt to znowu $\sqrt(n)$

\section{}

Wejście: Skierowany acykliczny graf.\\
Wyjście: Długość najdłuższej ścieżki.\\
LengthTo - tablica $|V(G)|$ elementów początkowo równych 0.\\
TopOrder(G) - posortowane topologicznie wierzchołki.\\


\begin{lstlisting}
    for each vertex V in topOrder(G) do
        for each edge (V, W) in E(G) do
            if LengthTo[W] <= LengthTo[V] + weight(G,(V,W)) then
               LengthTo[W]  = LengthTo[V] + weight(G,(V,W))
 
    return max(LengthTo[V] for V in V(G))
\end{lstlisting}

Sortowanie topologiczne działa w czasie $O(E+V)$, więc całość działa w czasie $O(E+V+E+V) = O(E+V)$.
Żeby wypisać drogę musimy tylko zapamiętywać, dla których wierzchołków spełniony był IF.


%%%%%%%%%%%%%%%%%%%%%%%%%%%%%%%%%%%%%%%%%%%%%%%%%%%%%%%%%%%%%%%%%%%%
\chapter{Lista 2}

%%%%%%%%%%%%%%%%%%%%%%%%%%%%%%%%%%%%%%%%%%%%%%%%%%%%%%%%%%%%%%%%%%%%
\chapter{Lista 3}

\section{} %1
Zadanie to polega na skonstruowaniu szybkiego algorytmu obliczania największego wspólnego dzielnika dwóch dodatnich liczb całkowitych $a$ i $b$. Przed podaniem algorytmu musimy udowodnić podane właściwości:

\begin{equation*}
gcd(a,b) =
 \begin{cases}
 2 \cdot gcd( \frac{a}{2}, \frac{b}{2}) & \text{a,b są parzyste;} \\
 gcd(a,\frac{b}{2}) & \text{a jest nieparzyste, b jest parzyste;} \\
 gcd(\frac{a-b}{2},b) & \text{a,b są nieparzyste.}
 \end{cases}
 \end{equation*}

\subsection{a)}
Jeżeli $a$ i $b$ są parzyste, $2$ na pewno jest ich wspólnym dizelnikiem. Jeżeli $a$ jest nieparzyste i $b$ jest parzyste, wiemy że $b$ dzieli się przez 2, a $a$ nie. Więc $gcd(a,b)$ pozostaje takie same dla $a$ i $b/2$. Ostatnia własność wynika z faktu, że dla nieparzystych $a$ i $b$, $(a-b)$ będzie parzyste.Ponieważ $gcd(a-b,b) = gcd(a,b)$ oraz $(a-b)$ jest teraz parzyste, możemy zastosować drugą własność.

\subsection{b) algorytm rekurencyjny}
\begin{lstlisting}
procedure gcd(a, b)
Input: Two n-bit integers a,b
Output: GCD of a and b
if a = b:
	return a
else if (a is even and b is even):
	return 2*gcd(a/2, b/2)
else if (a is odd and b is even):
	return gcd(a, b/2)
else if (a is odd and b is odd and a > b):
	return gcd((a-b)/2, b)
else if (a is odd and b is odd and a < b):
	return gcd(a,(b-a)/2)
\end{lstlisting}
\subsection{c) złożoność}
Założmy, że $a$ i $b$ są n-bitowymi liczbami. Rozmiar $a$ i $b$ wynosi $2n$ bitów. Wszystkie z czterech ifów, oprócz przypadku gdzie $a$ jest nieparzyste i $b$ jest parzyste, zmniejsza rozmiar $a$ i $b$ do $2n - 2$ bitów, gdzie wcześniej wymieniony przypadek zmniejsza ilość bitów do $2n-1$. Każda z operacji wykonuje się w czasie stałym ponieważ dzielimy lub mnożymy przez 2. Dla dwóch przypadków z odejmowaniem, mamy odejmowanie dwóch n-bitowych liczb (złożoność wynosi $c\cdot n$ gdzie n jest wielkością operandu). Zatem najgorszy przypadek czwartego ifa algorytmu przedstawimy jako:

\begin{equation*}
\begin{split}
T(2n) & = T(n-1) + cn \\
T(2n -1) & = T(2n - 2) + cn \\
T(2n - 2) & = T(2n - 3) + c(n-1)  \text{   oba operandy mają długość $n-1$} \\
T(2n - 3) & = T(2n - 4) + c(n-1) \\
... &  \\
T(2) &=  T(1) + c
\end{split}
\end{equation*}

Podstawieniami możemy zapisać $T(2n)$ jako:
$$ T(2n) = 2c\cdot \sum_{i=1}^n i $$
co daje nam $O(n^2)$ co w porównaniu do $O(n^3)$ czasu działania algorytmu euklidesa jest szybsze.
\section{} %2
\href{./solutions/lista_3_zadanie_2.pdf}{PDF}

\section{} %3
\section{} %4
\section{} %5
\subsection{a)}
W wektorze pamiętamy pierwszą kolumnę oraz pierwszy wiersz bez pierwszego wyrazu. Wektor ten ma rozmiar $2n - 1$, więc dodawanie dwóch takich wektorów mamy w $O(n)$.
\subsection{b)}
Macierz Toeplitza ma następującą postać blokową:
$$ T = \left[ \begin{matrix}
					A & B \\
					C & A
				\end{matrix} \right]$$

Zadanie polega na pomnożeniu macierzy T przez wektor blokowy $ T = \left[ \begin{matrix}x \\ y\end{matrix} \right] $ w czasie mniejszym niż $n^2$. Korzystając z dwóch (wzajemnie dualnych) obserwacji:

$$Ax = A(x + y - y) = A(x+y) - Ay$$

$$Ay = A(x + y - x) = A(x+y) - Ax$$


mamy:


\begin{equation*}
\begin{split}
T = \left[ \begin{matrix}A & B\\ C & A\end{matrix} \right] \cdot \left[ \begin{matrix} x \\ y\end{matrix} \right] = \left[ \begin{matrix} Ax + By\\ Cx + Ay\end{matrix} \right] = \\
= \left[ \begin{matrix}A(x+y)-Ay + By\\ Cx + Ay\end{matrix} \right] = \left[ \begin{matrix}A(x+y)+(B - A)y\\ Cx + Ay\end{matrix} \right] = \\
= \left[ \begin{matrix}A(x+y)+(B - A)y\\ Cx + A(x+y) - Ax \end{matrix} \right] =  \left[ \begin{matrix}A(x+y)+(B - A)y\\ A(x+y) + (C-A)x\end{matrix} \right]
\end{split}
\end{equation*}

I zauważamy, że złożoność czasowa to $T(n) = 3 T(n/2) + O(n)$ gdzie $O(n)$ zamyka sumę liniowych czasów wszystkich dodawań.

Rozwiązując typowe równanie rekurencyjne mamy $T(n) = O(n^{log_2 3}) < O(n^2)$.


\section{} %6
\section{} %7


%%%%%%%%%%%%%%%%%%%%%%%%%%%%%%%%%%%%%%%%%%%%%%%%%%%%%%%%%%%%%%%%%%%%
\chapter{Lista 4}

%%%%%%%%%%%%%%%%%%%%%%%%%%%%%%%%%%%%%%%%%%%%%%%%%%%%%%%%%%%%%%%%%%%%
\chapter{Lista 5}
\end{document}