\chapter{Lista 2}
\section{} %1

%%%%%%%%%%%%%%%%%%%%%%%%%%%%%%%
\section{} %2

Sortujemy odcinki rosnąco wg. $k_j$. Wybieramy pierwszy, potem kolejny, który się zmieści po pierwszym.\\

\noindent Dowód: Niech nasz algorytm daje uporządkowanie $U$. Załóżmy, że istnieje lepsze, optymalne uporządkowanie $S$. Weźmy pierwszą parę odcinków, które różnią się w obu tych uporzadkowaniach, tzn. $U_i \ne S_i$. Jeżeli odcinek $S_i$ kończy się wcześniej niż $U_i$ to zostałby on wybrany przez nasz algorytm. Jeśli kończy się na tej samej pozycji to nie ma znaczenia, który z nich został wybrany. Podobnie, indukcyjnie można zastosować to rozumowanie dla pozostałych par różnych odcinków w obu uporządkowaniach. Algorytm zachłanny daje więc rozwiązanie optymalne.

%%%%%%%%%%%%%%%%%%%%%%%%%%%%%%%
\section{} %3

Wytwarzanie ułamka egipskiego mniejszego od $\frac{x}{y}$ o najwiekszym mianowniku: 
\[\frac{x}{y} \rightarrow \frac{1}{\lceil \frac{y}{x} \rceil}\]

\noindent Licznik wyrażenia $\frac{x}{y} - \frac{1}{\lceil \frac{y}{x} \rceil} = \frac{x \lceil \frac{y}{x} \rceil - y}{x\lceil \frac{y}{x} \rceil}$ będzie malał, lecz zawsze będzie $\ge 0$:\\

\noindent Sprawdźmy, czy faktycznie $ 0 \leq x \lceil \frac{y}{x} \rceil - y < x$

\[x \lceil \frac{y}{x} \rceil - y < x \Leftrightarrow x \lceil \frac{y}{x} \rceil - x < y \Leftrightarrow \lceil \frac{y}{x} \rceil - 1 < \frac{y}{x} \textrm{.}\]

\noindent Co jest oczywiście prawdą. Dodatkowo: $x \lceil \frac{y}{x} \rceil - y \ge 0$. Więc ułamek zawsze się zmiejsza, i zatrzyma się na $0$.

\noindent Kontrprzykład na optymalność. Nasz da jakieś gówno, optymalny da:

\[\frac{5}{121}=\frac{1}{33}+\frac{1}{121}+\frac{1}{363}\]

%%%%%%%%%%%%%%%%%%%%%%%%%%%%%%%
\section{}%4

\subsection{Lemat 1}
\noindent W każdym momencie działania algorytmu, oraz po jego zakończeniu w $E'$ nie będzie cyklu.\\

\noindent Dowód: Załóżmy nie wprost, że podczas działania algorytmu w którymś etapie pojawiła się spójna składowa, w której jest cykl. Oznaczmy ją jako $S$. Rozważmy następujące sytuacje:\\

\begin{itemize}
\item $S$ powstała przez połączenie dwóch superwierzchołków $v_1$ i $v_2$. Oznacza to, że do zbioru $E'$ zostały dołączone krawędzie $e_i$ i $e_j$. Ponieważ $e_i$ została dołączona jako najlżejsza krawędź incydentna do $v_1$ więc $C(e_i) < C(e_j)$. Ale skoro $e_j$ została dołączona jako najlżejsza krawędź incydentna do $v_2$ to musi zachodzić  (Pamiętajmy, że w grafie nie ma krawędzi o takiej samej wadze) - sprzeczność.

\item $S$ powstała przez połączenie się trzech lub więcej superwierzchołków. Podzielmy powstały cykl $C$ na następujące części: niech $v_1,.../v_l$ będą kolejnymi superwierzchołkami należącymi do $C$ a $e_1,...,e_l$ będą kolejnymi krawędziami należącymi do $C$, które zostały dodane w zakończonym właśnie etapie algorytmu. W $C$ krawędzie $e_i$ oraz superwierzchołki $v_i$ występują na przemian. Z zasady działania algorytmu możemy stwierdzić, że aby powstał taki cykl, musi zachodzić $C(e_1) < C(e_2) ... < C(e_l) < C(e_1)$ - Sprzeczność.
\end{itemize}

\subsection{Lemat 2}

\noindent W każdym etapie działania algorytmu otrzymujemy dla każdego superwierzchołka minimalne drzewo rozpinające.\\

\noindent Dowód:

\begin{itemize}
\item Gdy zostanie zakończony etap 1:

Załóżmy, że istnieje taki superwierzchołek $v_i$, który nie jest minimalnym drzewem rozpinającym poddrzewa złożonego z wierzchołków należących do $v_i$. Weźmy więc takie minimalne drzewo rozpinające $T$. Istnieje krawędź $e_i$ taka, że $e_i \in E(v_i)$ oraz $e_i \not\in E(T)$ . Dodajmy $e_i$. W $T$ powstał cykl. Ponieważ $e_i$ jest incydenta do pewnego wierzchołka z tego cyklu, istnieje więc inna krawędź $e^\prime_i$ incydentna do tego samego wierzchołka. Jednak z tego, że $e_i \in E(v_i)$ wynika, że $C(e_i) < C(e^\prime_i)$. Jeśli usuniemy krawędź $e^\prime_i$ z $T$ otrzymamy mniejsze drzewo rozpinające, co jest sprzeczne z założeniem o minimalności $T$.

\item Gdy zostanie zakończony etap 2:

Z poprawności etapu 1 wiemy, że istnieje takie wywołanie etapu 2, w którym każdy z superwierzchołków jest minimalnym drzewem rozpinającym. Jest to choćby pierwsze wywołanie. Załóżmy zatem, że dla pewnego wywołania tego etapu otrzymano superwierzchołki będące minimalnymi drzewami rozpinającymi, jednak scaliło przynajmniej dwa z nich w taki sposób, że dało się otrzymać mniejsze drzewo rozpinające. Niech etap k-ty będzie pierwszym takim etapem, w którym coś się popsuło. Niech $E^\prime_1$ będzie zbiorem krawędzi przed wywołaniem etapu k, a $E^\prime_2$ będzie zbiorem krawędzi po jego wywołaniu. Niech $T$ będzie minimalnym drzewem rozpinającym takim, że $V(T) = V(v_i)$, ale że $E(T) \neq E(v_i)$ . Istnieje więc krawędź $e_i \in E(v_i)$ oraz $e_i \not\in E(T)$.\\

Fakt: Krawędź  dodana podczas k-tego wywołania. (Nie może należeć do $E^\prime_1$ gdyż inaczej superwierzchołek do którego by należała nie byłby minimalnym drzewem rozpinającym, co jest sprzeczne z dowodem dla pierwszego etapu i założeniem, że wywołanie k-te jest najmniejszym wywołaniem, które zwróciło nieoptymalne drzewa)
Dodajmy krawędź $e_i$ do $E(T)$. W $T$ powstał cykl. Ponieważ $e_i$ jest najmniejszą krawędzią incydentną do pewnego superwierzchołka z tego cyklu, istnieje więc inna krawędź incydenta do tego samego superwierzchołka. Jednak jej waga jest większa niż waga krawędzi $e_i$, zatem zastąpienie jej krawędzią $e_i$ da nam mniejsze drzewo rozpinające co jest sprzeczne z założeniem o optymalności T.
\end{itemize}

%%%%%%%%%%%%%%%%%%%%%%%%%%%%%%%

\section{} %5
\section{} %6
