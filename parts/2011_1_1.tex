%% POCZĄTEK: I termin 2011 (część 1)

\chapter{Egzamin 2011 część 1}

\section{zadanie 1}
Opisz algrotym mnożenia długich liczb (n bitowych)?

\section{zadanie 2}
Dlaczego porównania współrzędnych nie wystarczają do stiwerdzenia współliniowości punktów?

\section{zadanie 3} 
Jak posortwoać n liczb z przedziału $[1, n^2]$ w  czasie liniowym?

\section{zadaine 4}
Zapisz procedurę lączenia dwóch kopców dwumianowych typu eager.

\section{zadanie 5}


\section{zadanie 6}
Czemu algorytm magicznych trójek byłby gorszy od algorytmu magicznych piątek?

\section{zadanie 7}
Jak zwiększyć prawdopodobieństwo wylosowania "poprawnej" funkcji w trakcie konsturkcji słownika statycznego?

\section{zadanie 8}
Ile w pesymistycznym przpadku potrzeba rotacji do usunięcia węzła z AVL?

\section{zadanie 10}
Ile warstw musi mieć siec przelączników, żeby dalo się uzyskać wszystkei przesunięcia cykliczne ciągu n-elemntowego?

\chapter{Egzamin 2011 część 2}

\section{zadanie 1}
Danych jest $2 \cdot n$ punktów na okręgu.
\begin{description}
	\item[a)] Na ile sposobów można te punkty połączyć n nieprzecinającymi się odcinkami, takimi że każdy z punktów jest końcem dokadnie jednego odcinka.
	\item[b)] Ułużyc algorytm, który znajduje minimalną sumę długości odicnków łaczących (jak wyżej) te punktu.
\end{description}

\section{zadanie 2}
Dla danego k i tablicy kwadratwoej znaleźć liczbe dróg idących z "lewa" na "prawo", których koszt jest równy k.

\section{zadanie 3}
Mamy n dziewcząt i n chłopiąt. Chcemy tak ich dobrać w pary, by zminimalizować sumę różnic wzorów osób w parze.
Czy zachłanny algorytm - "dobierz w parę dziwczę i chłopie i najmniejszej różnicy wzrostu" zawsze zwraca optymalny wynik.
Jak tak, to napisz lepszy algorytm jak nie to napisz dlaczego?

\chapter{Egzamin 2011 część 3}

\section{zadanie 1}

\section{zadanie 2}

\section{zadanie 3} (niepamiętam fuck)
Dany jest n-elementowy ciąg liczb naturalnych. 
Niech D
Należy znaleść takie $i,j \in N$, że $i \leqslant j$.

%% KONIEC: I termin 2011 (część 1)