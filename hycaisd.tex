\documentclass[svgnames]{report}
\usepackage[utf8]{inputenc} 
\usepackage{polski}       
\usepackage{a4wide}
\usepackage{graphicx}
\usepackage{amsmath,amssymb}
\usepackage{bbm}             
\usepackage{amsthm}
\makeatletter
 \renewcommand\@seccntformat[1]{\csname  the#1\endcsname.\quad}
 
\usepackage{tikz}
\usepackage{kpfonts}
\usepackage[explicit]{titlesec}
\newcommand*\chapterlabel{}
\titleformat{\chapter}
  {\gdef\chapterlabel{}
   \normalfont\sffamily\Huge\bfseries\scshape}
  {\gdef\chapterlabel{\thechapter\ }}{0pt}
  {\begin{tikzpicture}[remember picture,overlay]
    \node[yshift=-3cm] at (current page.north west)
      {\begin{tikzpicture}[remember picture, overlay]
        \draw[fill=Gray] (0,0) rectangle
          (\paperwidth,3cm);
        \node[anchor=east,xshift=.9\paperwidth,rectangle,
              rounded corners=20pt,inner sep=11pt,
              fill=Black]
              {\color{white}\chapterlabel. #1};
       \end{tikzpicture}
      };
   \end{tikzpicture}
  }
\titlespacing*{\chapter}{0pt}{50pt}{-60pt}
 
\begin{document}
%\tableofcontents
\chapter{Wstęp}
O matko! Robimy aisd!
\chapter{poprawka 2010}
\section{zadanie 1}
\subsection{Treść}
Rozwiąż poniższe równanie:
\begin{equation*}
T(n) = 
	\begin{cases}
		c 					&	\hbox{jeśli n = 1}		\\
		2 \cdot T(n-1) + 1 	&	\hbox{jeśli $n > 1$}	\\
	\end{cases}
\end{equation*}
Czy funkcja $T(n)$ jest $O(n^{log_2 n})$?
\subsection{Rozwiązanie}
Rozwiązanie równania rekurencyjnego:
\begin{eqnarray*}
T(n) 	&=& 2 \cdot T(n-1) + 1	\\
		&=&	2 \cdot ( 2\cdot T(n-2) + 1 ) + 1	\\
		&=& 2^2 \cdot T(n-2) + 2 + 1	\\
		&=&	2^3 \cdot T(n-3) + 2^2 + 2 + 1 \\
		&...&	\\
		&=&	2^n\cdot T(1) + 2^{n-1} + ... + 2^2 + 2 + 1 \\
		&=& 2^n \cdot c + 2^n - 1	\\
\end{eqnarray*}
Pokażemy że $T(n) \not\in O(n^{log_2 n})$:

Z powyżej rozwiązanej rekurencji widać że: $T(n) \in \Omega (2^n)$, 
\begin{equation}
	\exists_{n_0 \in N} \forall_{n > n_0} n^{log_2 n} = \left(2^{log_2 n}\right)^{log_2 n} = 2^{(log_2 n)^2} < 2^n
\end{equation}
Wynika z tego, że $n^{log_2 n}$ nie jest ograniczeniem górnym.



\begin{thebibliography}{99}
\bibitem{Test} test reference
\end{thebibliography}
\end{document}
